\documentclass[]{article}
\usepackage{lmodern}
\usepackage{amssymb,amsmath}
\usepackage{ifxetex,ifluatex}
\usepackage{fixltx2e} % provides \textsubscript
\ifnum 0\ifxetex 1\fi\ifluatex 1\fi=0 % if pdftex
  \usepackage[T1]{fontenc}
  \usepackage[utf8]{inputenc}
\else % if luatex or xelatex
  \ifxetex
    \usepackage{mathspec}
    \usepackage{xltxtra}
    \setmainfont{Noto Sans CJK TC}
  \else
    \usepackage{fontspec}
  \fi
  \defaultfontfeatures{Ligatures=TeX,Scale=MatchLowercase}
\fi
% use upquote if available, for straight quotes in verbatim environments
\IfFileExists{upquote.sty}{\usepackage{upquote}}{}
% use microtype if available
\IfFileExists{microtype.sty}{%
\usepackage{microtype}
\UseMicrotypeSet[protrusion]{basicmath} % disable protrusion for tt fonts
}{}
\usepackage{hyperref}
\hypersetup{unicode=true,
            pdftitle={Algorithms Final Project Progress Report},
            pdfborder={0 0 0},
            breaklinks=true}
\urlstyle{same}  % don't use monospace font for urls
\IfFileExists{parskip.sty}{%
\usepackage{parskip}
}{% else
\setlength{\parindent}{0pt}
\setlength{\parskip}{6pt plus 2pt minus 1pt}
}
\setlength{\emergencystretch}{3em}  % prevent overfull lines
\providecommand{\tightlist}{%
  \setlength{\itemsep}{0pt}\setlength{\parskip}{0pt}}
\setcounter{secnumdepth}{0}
% Redefines (sub)paragraphs to behave more like sections
\ifx\paragraph\undefined\else
\let\oldparagraph\paragraph
\renewcommand{\paragraph}[1]{\oldparagraph{#1}\mbox{}}
\fi
\ifx\subparagraph\undefined\else
\let\oldsubparagraph\subparagraph
\renewcommand{\subparagraph}[1]{\oldsubparagraph{#1}\mbox{}}
\fi

\title{Algorithms Final Project Progress Report}
\date{May 24, 2016}

\begin{document}
\maketitle

\section{Team}\label{team}

Group 24

\begin{itemize}
\tightlist
\item
  劉典恆 b01901185
\item
  胡明衛 b01901133
\item
  許睿中 b01502119
\item
  張漢維 b01901181
\end{itemize}

\section{Problem}\label{problem}

\textbf{2016 D. Static Timing Analysis}

Find as many sensitizable paths as possible for combinational logic
circuit using shortest time.

\section{Progress}\label{progress}

\textbf{Documentation}

\begin{itemize}
\tightlist
\item
  {[}x{]} Summarize problem requirement. (@dianhenglau)
\item
  {[} {]} Summarize ideas of solving this problem. (@dianhenglau)
\item
  {[} {]} Read (and summarize if it is useful) about SAT. (@dianhenglau)
\item
  {[} {]} Summarize article provided by TA. (@raychunghsu)
\item
  {[} {]} Add compile requirement. (@dianhenglau)
\end{itemize}

\textbf{Classes}

\begin{itemize}
\tightlist
\item
  {[} {]} Define interface of Gate. (@davidhu34)
\item
  {[} {]} Define interface of Circuit. (@davidhu34)
\item
  {[} {]} Define interface of hash\_map. (@dianhenglau)
\end{itemize}

\textbf{Functions}

\begin{itemize}
\tightlist
\item
  {[}x{]} Write main function. (@dianhenglau)
\item
  {[} {]} Write a function to parse verilog netlist file into circuit.
  (@davidhu34)
\item
  {[}x{]} Write a function to output sensitizable paths according to
  requirement. (@davidhu34)
\item
  {[} {]} Write a function to find all sensitizable paths and their
  corresponding input vector. (@raychunghsu)
\item
  {[} {]} Write a function to verify whether a path is sensitizable
  path. This is for verification. (@raychunghsu)
\item
  {[} {]} Write string hash functions. (@dianhenglau)
\end{itemize}

\textbf{Other}

\begin{itemize}
\tightlist
\item
  {[} {]} Add OpenMP library to support multithread. (@dianhenglau)
\end{itemize}

\section{Ideas}\label{ideas}

\subsection{How to Find All Paths}\label{how-to-find-all-paths}

\begin{itemize}
\item
  Do depth first search from all input pins toward output pins.
\item
  Add the path to path list every time you reach an output pin.
\end{itemize}

\paragraph{Time Complexity}\label{time-complexity}

\begin{itemize}
\item
  Time complexity of doing depth first search on a single input pin is
  \(O(|E|)\) where \(|E|\) is number of edges of entire circuit. This is
  the worst case.
\item
  Time complexity of doing depth first search on all input pins is
  \(O(|PI| * |E|)\) where \(|PI|\) is number of input pins.
\end{itemize}

\subsection{How to Find All Paths Within
Constraint}\label{how-to-find-all-paths-within-constraint}

\begin{itemize}
\item
  Do depth first search from all input pins toward output pins.
\item
  While doing depth first search, calculate arrival time at each node.
  If arrival time is greater than or equal to time constraint, stop
  moving down this path.
\item
  Calculate slack everytime you reach output pin. If slack is greater
  than or equal to slack constraint, do not add this path into path
  list.
\end{itemize}

\paragraph{Time Complexity}\label{time-complexity-1}

\begin{itemize}
\tightlist
\item
  Time complexity is same as depth first search, i.e. \(O(|PI| * |E|)\).
\end{itemize}

\subsection{How to Calculate Arrival Time of All
Gates}\label{how-to-calculate-arrival-time-of-all-gates}

\begin{itemize}
\tightlist
\item
  From input pins, do breadth first search toward output pins.
\end{itemize}

\paragraph{Time Complexity}\label{time-complexity-2}

\begin{itemize}
\tightlist
\item
  \(O(|E|)\).
\end{itemize}

\subsection{How to Calculate Value of All
Points}\label{how-to-calculate-value-of-all-points}

Given an input vector and a set of nodes sorted according to each node's
arrival time, how to find value of all nodes.

\begin{itemize}
\tightlist
\item
  For each node in the sorted set, get its input nodes' value, then
  calculate its value.
\end{itemize}

\paragraph{Time Complexity}\label{time-complexity-3}

\begin{itemize}
\tightlist
\item
  Since each node need to access its input nodes, the overall time
  complexity should be \(O(|E|)\).
\end{itemize}

\subsection{How to Find Sensitizable
Paths}\label{how-to-find-sensitizable-paths}

\subsubsection{Method 1 (Brute Force)}\label{method-1-brute-force}

\begin{itemize}
\item
  Find all paths within constraint and collect them in a path list.
\item
  Try all possible permutation of input vectors.
\item
  For each input vector, assume circuit has become stable (ignoring
  delay), then find value of all points in the circuit.
\item
  With arrival time and value at each point, we can know whether a path
  is sensitizable. For each path in path list, check whether it is
  sensitizable.
\end{itemize}

\paragraph{Time Complexity}\label{time-complexity-4}

\begin{itemize}
\item
  Find all paths within constaint: \(O(|PI| * |E|)\).
\item
  Try all possible permutation: \(O(2^|PI|)\).
\item
  Find value of all points for an input vector: \(O(|E|)\).
\item
  Check all paths in path list: \(O(|P| * |p|)\) where \(|P|\) is number
  of paths and \(|p|\) is average number of node in a path.
\item
  Overall time complexity is
  \(O(|PI| * |E|) + O(2^|PI| * |E|) +  O(|P| * |p|)\), and
  \(O(2^|PI| * |E|)\) will probably dominate.
\end{itemize}

\paragraph{Note}\label{note}

\begin{itemize}
\tightlist
\item
  This method will find all sensitizable paths.
\end{itemize}

\subsubsection{Method 2 (Less Brute
Force)}\label{method-2-less-brute-force}

\begin{itemize}
\item
  Calculate arrival time of all gates.
\item
  Basically the idea is trace from output pins toward input pins. Try
  every possibility (condition) that make a path become a true path.
  Check whether our assumption has any contradiction.
\item
  Monitor slack constraint and time constraint while tracing.
\item
  Pseudo code:
\end{itemize}

TODO: Parallelize it; Monitor constraint; How to check for confliction?

\begin{verbatim}
sensitizable_paths = vector()
input_vecs = vector()

path = vector()

for po in output_pins
    trace(po.from)

trace(gate)
    path.add(pair(gate, gate.value))

    if gate.value == X
        assert(gate.type == PO)
        path.pop()
        gate.value = 0
        trace(gate)
        gate.value = 1
        trace(gate)
        gate.value = X

    else if gate.type == NAND
        # Try to make gate.from_a become a true path.

        ` start_code_block(basic_logic)
        if gate.from_a.arrival_time < gate.from_b.arrival_time
            if gate.from_a.value == X
                if gate.value == 1
                    gate.from_a.value = 0
                    trace(gate.from_a)
                    gate.from_a.value = X

        else if gate.from_a.arrival_time > gate.from_b.arrival_time
            if gate.from_b.value == X
                gate.from_b.value = 1
                gate.from_a.value = !gate.value
                trace(gate.from_a)
                gate.from_b.value = X
                gate.from_a.value = X

        else # Both of them have same arrival time.
            if gate.from_a.value == X
                if gate.value == 1
                    gate.from_a.value = 0
                    trace(gate.from_a)
                    gate.from_a.value = X

            if gate.from_b.value == X
                gate.from_b.value = 1
                gate.from_a.value = !gate.value
                trace(gate.from_a)
                gate.from_b.value = X
                gate.from_a.value = X
        ` end_code_block(basic_logic)

        # Try to make gate.from_b become a true path.

        ` print(swap("from_a", "from_b", basic_logic))

    else if gate.type == NOR
        # Try to make gate.from_a become a true path.

        ` print(swap("0", "1", basic_logic))

        # Try to make gate.from_b become a true path.

        ` print(swap("0", "1", swap("from_a", "from_b", basic_logic)))

    else if gate.type == NOT
        if gate.from.value == X
            gate.from.value = !gate.value
            trace(gate.from)
            gate.from.value = X

    else if gate.type == PI and no_conflict()
        sensitizable_paths.add(reverse(path))
        input_vec = vector()
        for pi in input_pins
            input_vec.add(pi.value)
        input_vecs.add(input_vec)
        
    else
        print("Error: Unknown gate type.\n")
        exit(1)

    path.pop_front()
\end{verbatim}

\paragraph{Time Complexity}\label{time-complexity-5}

\begin{itemize}
\item
  Calculate arrival time: \(O(|E|)\).
\item
  Tracing: ?
\end{itemize}

\paragraph{Note}\label{note-1}

\begin{itemize}
\item
  This method find all sensitizable paths too.
\item
  To adapt parallel execution, every thread must have their own copy of
  \texttt{gate.value}. Other gate attribute can be shared.
\end{itemize}

\end{document}
